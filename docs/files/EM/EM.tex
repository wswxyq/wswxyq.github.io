\documentclass[10pt, oneside]{article}   	% use "amsart" instead of "article" for AMSLaTeX format
\usepackage{geometry}                		% See geometry.pdf to learn the layout options. There are lots.
\geometry{letterpaper, margin=0.2in}                   		% ... or a4paper or a5paper or ... 
%\geometry{landscape}                		% Activate for rotated page geometry
%\usepackage[parfill]{parskip}    		% Activate to begin paragraphs with an empty line rather than an indent
\usepackage{graphicx}				% Use pdf, png, jpg, or eps§ with pdflatex; use eps in DVI mode
								% TeX will automatically convert eps --> pdf in pdflatex		
\usepackage{amssymb}
\usepackage{multicol,lipsum}
\usepackage{amsmath}
\usepackage{sectsty}
\usepackage[usenames, dvipsnames]{color}
\sectionfont{\fontsize{12}{15}\selectfont}
%SetFonts

%SetFonts
\usepackage{titling}
\setlength{\droptitle}{-5em}   % This is your set screw

\title{Jackson formula}
\author{Shaowei Wu}
%\date{}							% Activate to display a given date or no date

\begin{document}
\maketitle
%\section{}
%\subsection{}
\begin{multicols}{2}
\setlength{\columnseprule}{0.4pt}

\section{Lorentz transformation}


\subsection{$K'$ moving at speed $v$ along $z$ direction viewed by $K$}
$$
\left.\begin{array}{l}
  x'_0=\gamma (x_0- \beta x_1)\\
  x'_1=\gamma (x_1- \beta x_0)\\
  x'_2=x_2\\
  x'_3=x_3\\
\end{array}\right\}(11.16)
$$
where $\beta=|\boldsymbol{\beta}|=v/c$, $\gamma=\frac{1}{\sqrt{1-\beta ^2}}$\quad(11.17)\\
Inverse Lorentz transformation
$$
\left.\begin{array}{l}
x_{0}=\gamma\left(x_{0}^{\prime}+\beta x_{1}^{\prime}\right) \\ 
x_{1}=\gamma\left(x_{1}^{\prime}+\beta x_{0}^{\prime}\right) \\ 
x_{2}=x_{2}^{\prime} \\
x_{3}=x_{3}^{\prime}
\end{array}\right\}(11.18)
$$
Generally, 
$$
\left.\begin{array}{l}
x_{0}^{\prime}=\gamma\left(x_{0}-\boldsymbol{\beta} \cdot \mathbf{x}\right) \\ \\
\mathbf{x}^{\prime}=\mathbf{x}+\frac{(\gamma-1)}{\beta^{2}}(\boldsymbol{\beta} \cdot \mathbf{x}) \boldsymbol{\beta}-\gamma \boldsymbol{\beta} x_{0}
\end{array}\right\}(11.19)
$$


\subsection{4-vectors}
$$(A_0,\boldsymbol{A})=(A_0, A_1, A_2, A_3)$$
$$
\left.\begin{array}{rl}
A_{0}^{\prime} & =\gamma\left(A_{0}-\boldsymbol{\beta} \cdot \mathbf{A}\right) \\ A_{\|}^{\prime} & =\gamma\left(A_{\|}-\beta A_{0}\right) \\ 
\mathbf{A}_{\perp}^{\prime} & =\mathbf{A}_{\perp}
\end{array}\right\}(11.22)
$$
$\perp$ and $\|$ mean perpendicular and parallel to $\boldsymbol{v}$.\\
The scalar product of two 4-vectors are invariant.\\
Notation: $(x_0, \boldsymbol{x})\equiv (x_0=c t, x_1=z, x_2=x, x_3=y)$

\subsection{proper time $\tau$}
$$d \tau=d t \sqrt{1-\beta^{2}(t)}=\frac{d t}{\gamma(t)}$$
(11.26)\\
$$t_{2}-t_{1}=\int_{\tau_{1}}^{\tau_{2}} \frac{d \tau}{\sqrt{1-\beta^{2}(\tau)}}=\int_{\tau_{1}}^{\tau_{2}} \gamma(\tau) d \tau$$
(11.27)\\
\\
Timelike:\\
$$c^{2}\left(t_{1}-t_{2}\right)^{2}-\left|\mathbf{x}_{1}-\mathbf{x}_{2}\right|^{2}>0$$
Spacelike:\\
$$c^{2}\left(t_{1}-t_{2}\right)^{2}-\left|\mathbf{x}_{1}-\mathbf{x}_{2}\right|^{2}<0$$
Lightlike:\\
$$c^{2}\left(t_{1}-t_{2}\right)^{2}-\left|\mathbf{x}_{1}-\mathbf{x}_{2}\right|^{2}=0$$

\subsection{relativistic Doppler shift}
phase of a wave $\phi=\omega t - \boldsymbol{k} \cdot \boldsymbol{x}$ is invariant.\\
$(k_0, \boldsymbol{k})\equiv(\omega/c, \boldsymbol{k})$ is a 4-vector.\\
For light waves: $k_0=|\boldsymbol{k}|$. Thus, $\omega'=\gamma\omega(1-\beta cos\theta)\quad(11.30)$

\section{addition of velocity}
A point P moves in inertia frame K' with velocity $\boldsymbol{u}\equiv\frac{d \mathbf{x'}}{dt'}$. K' moves in inertia frame K with velocity $\boldsymbol{v}$. What is the velocity of P in K?
$$\begin{aligned} 
u_{\|}=& \frac{u_{\|}^{\prime}+v}{1+\frac{\mathbf{v} \cdot \mathbf{u}^{\prime}}{c^{2}}} \\ 
\mathbf{u}_{\perp}=& \frac{\mathbf{u}_{\perp}^{\prime}}{\gamma_{v}\left(1+\frac{\mathbf{v} \cdot \mathbf{u}^{\prime}}{c^{2}}\right)} 
\end{aligned}\quad(11.31)$$
If $\boldsymbol{u}$ and $\boldsymbol{v}$ are parallel, then:\\
$$u=\frac{u^{\prime}+v}{1+\frac{v u^{\prime}}{c^{2}}}\quad(11.33)$$

\section{relativistic dynamics}
\subsection{4-velocity $(U_0, \mathbf{U})=(\gamma_{u} c, \gamma_{u} \mathbf{u})$}
$$\left.\begin{array}{rl}
U_{0} & \equiv \frac{d x_{0}}{d \tau}=\frac{d x_{0}}{d t} \frac{d t}{d \tau}=\gamma_{u} c \\\\
\mathbf{U} & \equiv \frac{d \mathbf{x}}{d \tau}=\frac{d \mathbf{x}}{d t} \frac{d t}{d \tau}=\gamma_{u} \mathbf{u}
\end{array}\right\}(11.36)$$

\subsection{4-momentum $(p_0, \mathbf{p})=(E/c, \gamma_u m \mathbf{u})=m(U_0, \mathbf{U})$}
The only way to construct a 4-vector of momentum is to multiply 4-velocity by rest mass $m$.\\
The momentum of a particle of mass $m$ and velocity $\mathbf{u}$ is
$$\mathbf{p}=\gamma_u m \mathbf{u}=\frac{m \mathbf{u}}{\sqrt{1-\frac{u^2}{c^2}}}\quad(11.46)$$ 
The total energy of a particle of mass $m$ is $$E=\gamma_u m c^2=\frac{m c^2}{\sqrt{1-\frac{u^2}{c^2}}}\quad (11.51)$$
Kinetic energy: $T(u)=mc^2(\frac{1}{\sqrt{1-\frac{u^2}{c^2}}}-1)$\quad(11.49)\\
The conservation of total energy and total momentum can be written as conservation of total 4-momentum.
We also have:\\
$$E=\sqrt{c^2 p^2+m^2 c^4}\quad(11.55)$$

\subsection{rapidity/boost parameter $\xi$}
$$\beta=tanh \xi$$
$$\gamma=cosh\xi$$
$$\gamma\beta=sinh\xi$$
(11.20)

\section{homogeneous Lorentz group/Poincare group}
Mathematical equations expressing the law of nature must be covariant/inariant in form.
\subsection{tensor of rank $k$}
Transformation in space time point: $x\rightarrow x'$

rank 0: scalar (invariant) \\  
\\
rank 1: contravariant vector\\
$$A'^\alpha=\frac{\partial x'^\alpha}{\partial x^\beta}A^\beta \quad (11.61)$$
rank 1: covariant vector\\
$$A'_\alpha=\frac{\partial x^\beta}{\partial x'^\alpha}A_\beta \quad (11.62)$$
rank-2 contravariant tensor\\
$$F'^{\alpha \beta}=\frac{\partial x'^{\alpha}}{\partial x^{\gamma}} \frac{\partial x'^{\beta}}{\partial x^{\delta}} F^{\gamma \delta} \quad(11.63)$$
rank-2 covariant tensor\\
$$F'_{\alpha \beta}=\frac{\partial x^{\gamma}}{\partial x'^{\alpha}} \frac{\partial x^{\delta}}{\partial x'^{\beta}} F_{\gamma \delta} \quad(11.64)$$
inner/scalar product of two vectors:
$$B\cdot A\equiv B_\alpha A^\alpha \quad (11.65)$$
\subsection{metric tensor}
$$g_{\alpha\beta}=
\begin{pmatrix} 
1 & 0 & 0 & 0 \\
0 & -1 & 0 & 0 \\
0 & 0 & -1 & 0 \\
0 & 0 & 0 & -1
\end{pmatrix}=g^{\alpha\beta} \quad(11.81)$$

$$x_\alpha=g_{\alpha\beta}x^\beta \quad(11.72)$$
$$x^\alpha=g^{\alpha\beta}x_\beta \quad(11.73)$$

Concisely we have:
$$A^{\alpha}=\left(A^{0}, \mathbf{A}\right), \quad A_{\alpha}=\left(A^{0},-\mathbf{A}\right)\quad (11.75)$$
$$\begin{aligned} \partial^{\alpha} & \equiv \frac{\partial}{\partial x_{\alpha}}=\left(\frac{\partial}{\partial x^{0}},-\nabla\right) \\
\partial_{\alpha} & \equiv \frac{\partial}{\partial x^{\alpha}}=\left(\frac{\partial}{\partial x^{0}}, \nabla\right) \end{aligned}\\
\quad(11.76) 
$$
$$\partial^{\alpha} A_{\alpha}=\partial_{\alpha} A^{\alpha}=\frac{\partial A^{0}}{\partial x^{0}}+\nabla \cdot \mathbf{A}\quad (11.77)$$

Laplacian operator:\\
$$\square \equiv \partial_{\alpha} \partial^{\alpha}=\frac{\partial^{2}}{\partial x^{02}}-\nabla^{2}\quad(11.78)$$

\section{relativistic electrodynamics}
\subsection{Lorentz force equation}
$$\textcolor{red}{\frac{d \mathbf{p}}{d t}=q\left(\mathbf{E}+\frac{\mathbf{v}}{c} \times \mathbf{B}\right)}\quad(11.124)$$
If we use proper time instead, then\\
$$\begin{aligned}
\frac{d \mathbf{p}}{d \tau}=&\frac{d \mathbf{p}}{d t}\frac{d t}{d \tau}\\
=&q\gamma \left(\mathbf{E}+\frac{\mathbf{v}}{c} \times \mathbf{B}\right)\\
=&\frac{q}{c}(\gamma c \mathbf{E}+\gamma \mathbf{v}\times B)\\
=&\frac{q}{c}\left(U_{0} \mathbf{E}+\mathbf{U} \times \mathbf{B}\right)
\end{aligned}\quad(11.125)$$
Similar to Newton dynamic, we have:
$$\frac{d p_{0}}{d \tau}=\frac{q}{c} \mathbf{U} \cdot \mathbf{E}\quad(11.126)$$

\subsection{continuity equation}
Electric charge is conserved. The continuity equation still works.
$$\textcolor{red}{\frac{\partial \rho}{\partial t}+\nabla \cdot \mathbf{J}=0} \quad (11.127)$$
Equation (11.77) and (11.127) implies that we can write charge density $\rho$ and current density $J$ in 4-vector form:\\
$$J^\alpha=(c \rho, \mathbf{J})\quad(11.128)$$
Then we have:\\
$$\partial_{\alpha} J^{\alpha}=0\quad (11.129)$$

\subsection{Lorentz gauge}
$$\begin{array}{l}\frac{1}{c^{2}} \frac{\partial^{2} \mathbf{A}}{\partial t^{2}}-\nabla^{2} \mathbf{A}=\frac{4 \pi}{c} \mathbf{J} \\
\\
\frac{1}{c^{2}} \frac{\partial^{2} \Phi}{\partial t^{2}}-\nabla^{2} \Phi=4 \pi \rho
\end{array}\quad(11.130)$$
\\
Lorentz condition:\\
$$\frac{1}{c} \frac{\partial \Phi}{\partial t}+\mathbf{\nabla} \cdot \mathbf{A}=0\quad(11.131)$$

$$A^\alpha=(\Phi, \mathbf{A})\quad(11.132)$$
Equation (11.130)+(11.78):\\
$$\square A^{\alpha}=\frac{4 \pi}{c} J^{\alpha}\quad(11.133)$$
Equation (11.131)+(11.77):\\
$$\partial_{\alpha} A^{\alpha}=0\quad(11.133)$$
Field in terms of potential:\\
$$\begin{aligned} \mathbf{E} &=-\frac{1}{c} \frac{\partial \mathbf{A}}{\partial t}-\nabla \Phi \\ \mathbf{B} &=\mathbf{\nabla} \times \mathbf{A} \end{aligned}\quad(11.134)$$

$$F^{\alpha \beta}=\left(\begin{array}{cccc}0 & -E_{x} & -E_{y} & -E_{z} \\ E_{x} & 0 & -B_{z} & B_{y} \\ E_{y} & B_{z} & 0 & -B_{x} \\ E_{z} & -B_{y} & B_{x} & 0\end{array}\right)$$
$$\frac{d p^{\alpha}}{d \tau}=m \frac{d U^{\alpha}}{d \tau}=\frac{q}{c} F^{\alpha \beta} U_{\beta}$$

\subsection{transformation of electromagnetic field}
Transformation of the rank-2 tensor $F^{\alpha \beta}$ follows equation (11.63).
Thus we can deduce the transformation of field directly as below.
$$\begin{array}{ll}E_{1}^{\prime}=E_{1} & B_{1}^{\prime}=B_{1} \\ E_{2}^{\prime}=\gamma\left(E_{2}-\beta B_{3}\right) & B_{2}^{\prime}=\gamma\left(B_{2}+\beta E_{3}\right) \\ E_{3}^{\prime}=\gamma\left(E_{3}+\beta B_{2}\right) & B_{3}^{\prime}=\gamma\left(B_{3}-\beta E_{2}\right)\end{array}\quad(11.148)$$

$$\begin{array}{l}\mathbf{E}^{\prime}=\gamma(\mathbf{E}+\boldsymbol{\beta} \times \mathbf{B})-\frac{\gamma^{2}}{\gamma+1} \boldsymbol{\beta}(\boldsymbol{\beta} \cdot \mathbf{E}) \\ \mathbf{B}^{\prime}=\gamma(\mathbf{B}-\boldsymbol{\beta} \times \mathbf{E})-\frac{\gamma^{2}}{\gamma+1} \boldsymbol{\beta}(\boldsymbol{\beta} \cdot \mathbf{B})\end{array}\quad(11.149)$$
\\
For electromagnetic field transformation of a charged particle, see Jackson 11.152.





\end{multicols}

\end{document}  